%!TEX TS-program = pdflatexmk
%!TEX root = manual.tex

% Copyright (c) 2018 - 2023, Martin Scheidt (ISC license)
% Permission to use, copy, modify, and/or distribute this file for any purpose with or without fee is hereby granted, provided that the above copyright notice and this permission notice appear in all copies.

\begin{versionhistory}
  %\vhEntry{<Version>}{<Date>}{<Author1>|<Author2>|...}{<Changes>}
  \vhEntry{0.1}{2018-09-14}{MS}{
    Basic concept of a library with railway topology symbols and some examples.
  }
  \vhEntry{0.2}{2018-12-19}{MS}{
    Added transmitters and minor improvements.
  }
  \vhEntry{0.3}{2019-04-04}{MS}{
    Moved snippet folder to root folder and defined and used color foreground and background.
  }
  \vhEntry{0.4}{2019-07-21}{MS}{
    Reworked library for common tikz library layout.
  }
  \vhEntry{0.5}{2020-01-14}{MS}{
    Introducing new syntax and providing a documentation.
  }
  \vhEntry{0.5.1}{2020-02-10}{MS}{
    Modified symbol "end of movement authority"; added symbols "braking point" and "danger point".
  }
  \vhEntry{0.6}{2021-01-02}{MS}{
    Added symbols for "direction control", "track marking", "pylons" and electric wiring; changed symbol for "friction bufferstop"; created an encapsulating package for future flexibility - changed load command for library to \textbackslash usepackage\{tikz-trackschematic\}.
  }
  \vhEntry{0.6.1}{2021-09-30}{MS}{
    removed package requirement lmodern, minor correction in manual, added citation information
  }
  \vhEntry{0.6.2}{2021-10-15}{MS}{
    bug fixing
  }
  \vhEntry{0.6.3}{2022-02-15}{MS|GW}{
    fixed spelling error and documented (slip-) turnout option: points=moving; updated link to signalschablone; automated testing and releasing
  }
  \vhEntry{0.7.0}{2022-04-02}{MS|GW}{
    revised symbol and syntax for balises; replaced "\textbackslash gettikzxy" with "\textbackslash path let" syntax; fixed PackageWarning Error in development mode; fixed foreground of sidetrack (alias)
  }
  \vhEntry{0.7.1}{2022-06-02}{MS}{
    handeling color background and foreground with native xcolor alias "\textbackslash colorlet{}{}" instead of pgf macro
  }
  \vhEntry{0.8.0}{2023-05-05}{MS}{
    new symbol "trigger point"
  }
\end{versionhistory}
