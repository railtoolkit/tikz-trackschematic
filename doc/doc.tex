%% symbol library for TikZ track schematics
%
% Copyright 2018,2019 Martin Scheidt (ISC license)

% Permission to use, copy, modify, and/or distribute this file for any purpose with or without fee is hereby granted, provided that the above copyright notice and this permission notice appear in all copies.

\documentclass[
  draft,
  paper=a4,
  version=3.25,
  pagesize=pdftex,
  twoside=false,
  toc=listof,
]{scrartcl}
% --------[  Coding and Language  ]----------
\usepackage{scrhack}
\usepackage[utf8]{inputenc}
\usepackage[T1]{fontenc}
\usepackage[main=english]{babel}
% --------[   revision history    ]----------
\usepackage[tocentry]{vhistory}
\input{authors.tex}
% --------[ Layout  ]-----------
\pretolerance=8000
\tolerance=9500
\hbadness=8000
\vbadness=10000
\displaywidowpenalty=10000
\clubpenalty=10000
\widowpenalty=10000
\usepackage{lmodern,microtype,mathptmx,courier}
\usepackage[scaled=0.92]{helvet}
\usepackage[
  automark,
  headsepline,
  draft=false
]{scrlayer-scrpage}
\pagestyle{scrheadings}
% -----------[ PDF linking ]----------------
\usepackage[
  pdftex,
  pdfpagelabels, % modify PDF page labels
  hyperindex,
  hyperfigures,
  bookmarksopen,
  bookmarksnumbered,
  draft=false,
  pageanchor=true, % Determines whether every page is given an implicit anchor at the top left corner
  %pagebackref, % Adds ‘backlink’ text to the end of each item in the bibliography, as a list of page numbers
  %linktocpage, % make page number, not text, be link on TOC, LOF and LOT
  breaklinks=true, % allow links to break over lines by making links over multiple lines into PDF links to the same target
  colorlinks=true, % Colors the text of links and anchors
  linkcolor=base01, % Color for normal internal links
  urlcolor=blue, % Color for web links
]{hyperref} % PDF with a linked TableOfContent
\usepackage{bookmark} % Adding package bookmark improves bookmarks handling.
\usepackage{url}
% -------[ PDF Informations ]---------
\hypersetup{%
  pdftitle={tikz/trackschematic},
  pdfsubject={A tikz toolbox for track schematics},
  pdfauthor={Martin Scheidt},
  pdfkeywords={latex, tikz, library, railway, track, layout}
}

\usepackage[inline]{enumitem}
\usepackage{tikz}

\usepackage[prefix=]{xcolor-solarized}
\def\rootTrackschematic{../../tikz-trackschematic}
\def\srcTrackschematic{\rootTrackschematic/src/tikzlibrarytrackschematic}
\input{\srcTrackschematic.topology.code.tex}
\input{\srcTrackschematic.trafficcontrol.code.tex}
\input{\srcTrackschematic.vehicles.code.tex}
\input{\srcTrackschematic.constructions.code.tex}

\begin{document}

\title{\tikz\node[scale=1.2]{\color{gray}\Huge\sffamily \{\textcolor{black}{Ti\textcolor{orange}{\emph{k}}Z}/\textcolor{blue}{trackschematic}\}};}
\subtitle{A Ti\emph{k}Z library for track schematics}
\author{\vhListAllAuthorsLong}
\date{Version \vhCurrentVersion~ from \vhCurrentDate}

\maketitle

\tableofcontents

\section{Introduction}\label{sec:intro}

\subsection[About]{About tikz/trackschematic}

  The Ti\emph{k}Z \emph{trackschematic} library is a toolbox of symbols geared primarily towards creating track schematic for either research or educational purposes.
  It provides a tikz frontends to some of the symbols which maybe needed to describe situations and layouts in railway operation.
  The library is divided into four sublibraries:
  \begin{itemize*}[label={}]
    \item topology,
    \item traffic control,
    \item vehicles, and
    \item constructions.
  \end{itemize*}


\subsection{Requirements}\label{sec:require}

  The library uses Ti\emph{k}Z and it is based the following packages:
  \begin{itemize*}[label={}]
    \item tikz,
    \item xcolor, and
    \item etoolbox.
  \end{itemize*}
  Further more it uses the following Ti\emph{k}Z libraries:
  \begin{itemize*}[label={}]
    \item calc,
    \item patterns, and
    \item arrows.meta.
  \end{itemize*}


\subsection{License}

  Copyright 2018, 2019 \MS. Permission to use, copy, modify, and/or distribute this file for any purpose with or without fee is hereby granted, provided that the above copyright notice and this permission notice appear in all copies (\href{https://www.tldrlegal.com/l/isc}{ISC license}).

\section{Usage}\label{sec:use}

  loading ther library

% \appendix
%!TEX TS-program = pdflatexmk
%!TEX root = manual.tex

% Copyright (c) 2018 - 2022, Martin Scheidt (ISC license)
% Permission to use, copy, modify, and/or distribute this file for any purpose with or without fee is hereby granted, provided that the above copyright notice and this permission notice appear in all copies.

\begin{versionhistory}
  %\vhEntry{<Version>}{<Date>}{<Author1>|<Author2>|...}{<Changes>}
  \vhEntry{0.1}{2018-09-14}{MS}{
    Basic concept of a library with railway topology symbols and some examples.
  }
  \vhEntry{0.2}{2018-12-19}{MS}{
    Added transmitters and minor improvements.
  }
  \vhEntry{0.3}{2019-04-04}{MS}{
    Moved snippet folder to root folder and defined and used color foreground and background.
  }
  \vhEntry{0.4}{2019-07-21}{MS}{
    Reworked library for common tikz library layout.
  }
  \vhEntry{0.5}{2020-01-14}{MS}{
    Introducing new syntax and providing a documentation.
  }
  \vhEntry{0.5.1}{2020-02-10}{MS}{
    Modified symbol "end of movement authority"; added symbols "braking point" and "danger point".
  }
  \vhEntry{0.6}{2021-01-02}{MS}{
    Added symbols for "direction control", "track marking", "pylons" and electric wiring; changed symbol for "friction bufferstop"; created an encapsulating package for future flexibility - changed load command for library to \textbackslash usepackage\{tikz-trackschematic\}.
  }
  \vhEntry{0.6.1}{2021-09-30}{MS}{
    removed package requirement lmodern, minor correction in manual, added citation information
  }
  \vhEntry{0.6.2}{2021-10-15}{MS}{
    bug fixing
  }
  \vhEntry{0.6.3}{2022-02-15}{MS|GW}{
    fixed spelling error and documented (slip-)turnout option: points=moving; updated link to signalschablone; automated testing and releasing
  }
\end{versionhistory}

\vhListAllAuthorsLongWithAbbrev
\end{document}